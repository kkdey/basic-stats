%<<echo=FALSE>>=
%OLD <- options(width=90)
%@
%<<echo=FALSE>>=
%options(OLD) 
%@

\documentclass{beamer}\usepackage[]{graphicx}\usepackage[]{color}
%% maxwidth is the original width if it is less than linewidth
%% otherwise use linewidth (to make sure the graphics do not exceed the margin)
\makeatletter
\def\maxwidth{ %
  \ifdim\Gin@nat@width>\linewidth
    \linewidth
  \else
    \Gin@nat@width
  \fi
}
\makeatother

\definecolor{fgcolor}{rgb}{0.102, 0.102, 0.102}
\newcommand{\hlnum}[1]{\textcolor[rgb]{0.2,0.2,0.2}{#1}}%
\newcommand{\hlstr}[1]{\textcolor[rgb]{0.2,0.2,0.2}{#1}}%
\newcommand{\hlcom}[1]{\textcolor[rgb]{0.302,0.302,0.302}{\textit{#1}}}%
\newcommand{\hlopt}[1]{\textcolor[rgb]{0.102,0.102,0.102}{#1}}%
\newcommand{\hlstd}[1]{\textcolor[rgb]{0.102,0.102,0.102}{#1}}%
\newcommand{\hlkwa}[1]{\textcolor[rgb]{0.102,0.102,0.102}{#1}}%
\newcommand{\hlkwb}[1]{\textcolor[rgb]{0.102,0.102,0.102}{#1}}%
\newcommand{\hlkwc}[1]{\textcolor[rgb]{0.2,0.2,0.2}{#1}}%
\newcommand{\hlkwd}[1]{\textcolor[rgb]{0.102,0.102,0.102}{\textbf{#1}}}%

\usepackage{framed}
\makeatletter
\newenvironment{kframe}{%
 \def\at@end@of@kframe{}%
 \ifinner\ifhmode%
  \def\at@end@of@kframe{\end{minipage}}%
  \begin{minipage}{\columnwidth}%
 \fi\fi%
 \def\FrameCommand##1{\hskip\@totalleftmargin \hskip-\fboxsep
 \colorbox{shadecolor}{##1}\hskip-\fboxsep
     % There is no \\@totalrightmargin, so:
     \hskip-\linewidth \hskip-\@totalleftmargin \hskip\columnwidth}%
 \MakeFramed {\advance\hsize-\width
   \@totalleftmargin\z@ \linewidth\hsize
   \@setminipage}}%
 {\par\unskip\endMakeFramed%
 \at@end@of@kframe}
\makeatother

\definecolor{shadecolor}{rgb}{.97, .97, .97}
\definecolor{messagecolor}{rgb}{0, 0, 0}
\definecolor{warningcolor}{rgb}{1, 0, 1}
\definecolor{errorcolor}{rgb}{1, 0, 0}
\newenvironment{knitrout}{}{} % an empty environment to be redefined in TeX

\usepackage{alltt}% regular slides (with pauses)
%\documentclass[handout]{beamer}% handout (no pauses)

%%%%%%%%%%%%%%%%%%%%%%%%%%%%%%%%%%%%%%%%%%%%%%%%%%%%%%%%%%%%%%%%%%%%%%%%%
%%%%%%% Change the lecture information here %%%%%%%%%%%%%%%%
\def\chapnum{Week \#5}
\title{STAT234: Lecture 4 - Sums of random Variables}
\author{Kushal K. Dey}
\date{}
%%%%%%%%%%%%%%%%%%%%%%%%%%%%%%%%%%%%%%%%%%%%%%%%%%%%%%%%%%%%%%%%%%%%%%%%%

%%%%%% Start of suggested definitions and packages %%%%%%%%%%%%
%%%%%% Do not change unless you really know what you are doing %%%%%%%%%%
%%%%%%%%%%%%%%%%%%%%%%%%%%%%%%%%%%%%%%%%%%%%%%%%%%%%%%%%%%%%%%%%%%%%%%%%%

\usepackage{enumerate}
\usepackage{amsmath, bbm}
\usepackage[misc]{ifsym} % for the dice symbol \Cube{}
\usepackage[latin1]{inputenc}
\usepackage{hyperref}

%\usepackage{comment}
%\usepackage{pstricks}
%\usepackage{graphicx}
%\usepackage{booktabs}
%\usepackage{pgfpages}
%\pgfpagesuselayout{2 on 1}[a4paper,border shrink=3mm]
%\pgfpagesuselayout{4 on 1}[a4paper,landscape,border shrink=3mm

\usepackage{setspace}
\ifdefined\knitrout
  \renewenvironment{knitrout}{\begin{spacing}{0.75}\begin{tiny}}{\end{tiny}\end{spacing}}
\else
\fi

%%%%%%%%%%%%%%% Defined Shortcuts (macros) %%%%%%%%%%%%%
% parameters and statistics
\newcommand{\xbar}{\overline{x}}
\newcommand{\Xbar}{\overline{X}}
\newcommand{\ybar}{\overline{y}}
\newcommand{\Ybar}{\overline{Y}}
\newcommand{\dbar}{\overline{d}}
\newcommand{\Dbar}{\overline{D}}
\newcommand{\zbar}{\overline{z}}
\newcommand{\Zbar}{\overline{Z}}
\newcommand{\ehat}{\widehat{\epsilon}}
\newcommand{\yhat}{\widehat{y}}
\newcommand{\Yhat}{\widehat{Y}}
\newcommand{\betaa}{{\beta_0}}
\newcommand{\betab}{{\beta_1}}
\newcommand{\betac}{{\beta_2}}
\newcommand{\betad}{{\beta_3}}
\newcommand{\BETA}{{\boldsymbol\beta}}
\newcommand{\betahata}{\widehat{\beta_0}}
\newcommand{\betahatb}{\widehat{\beta_1}}
\newcommand{\betahatc}{\widehat{\beta_2}}
\newcommand{\betahatd}{\widehat{\beta_3}}
\newcommand{\bhat}{\widehat{b}}
\newcommand{\btilde}{\widetilde{b}}
\newcommand{\ahat}{\widehat{a}}
\newcommand{\atilde}{\widetilde{a}}
\newcommand{\rss}{\mathit{SSE}}
\newcommand{\sigmahat}{\widehat{\sigma}}
\newcommand{\betahat}{\widehat{\beta}}
\newcommand{\thetahat}{\widehat{\theta}}
\newcommand{\phat}{\widehat{p}}
\newcommand{\pihat}{\widehat{\pi}}
\newcommand{\muhat}{\widehat{\mu}}
% real numbers and integers
\newcommand{\reals}{\mathbbm{R}}
\newcommand{\integers}{\mathbbm{N}}
%distributions
\newcommand{\normal}{\textsf{Norm}}
\newcommand{\Bin}{\textsf{Binom}}
\newcommand{\Uni}{\textsf{Unif}}
\newcommand{\Poisson}{\textsf{Pois}}
\newcommand{\Exp}{\textsf{Exp}}
\newcommand{\Beta}{\textsf{Beta}}
\newcommand{\iid}{\stackrel{\mathrm{iid}}{\sim}}
% probability and expected value
\newcommand{\rv}{r.v.\ }
\newcommand{\prob}{{\rm P}}
\newcommand{\mean}{\mathrm{E}}
\newcommand{\var}{\mathrm{Var}}
\newcommand{\Var}{\mathrm{Var}}
\newcommand{\cov}{\mathrm{Cov}}
\newcommand{\corr}{\mathop{\mathrm{Corr}}}
% measures of spread
\newcommand{\IQR}{\textit{IQR}}
\newcommand{\SAD}{\textit{SAD}}
\newcommand{\MAD}{\textit{MAD}}
\newcommand{\SSD}{\textit{SSD}}
\newcommand{\MSD}{\textit{MSD}}
\newcommand{\RMSD}{\textit{RMSD}}
\newcommand{\MSE}{\textit{MSE}}
\newcommand{\MSR}{\textit{MSR}}
% formatting code and such
\providecommand{\variable}[1]{}
\renewcommand{\variable}[1]{{\color{green!50!black}\texttt{#1}}}
\providecommand{\function}[1]{}
\renewcommand{\function}[1]{{\color{purple!75!blue}\texttt{\StrSubstitute{#1}{()}{}()}}}
\providecommand{\option}[1]{}
\renewcommand{\option}[1]{{\color{brown!80!black}\texttt{#1}}}
\providecommand{\pkg}[1]{}
\renewcommand{\pkg}[1]{{\color{red!80!black}\texttt{#1}}}
\providecommand{\code}[1]{}
\renewcommand{\code}[1]{{\color{blue!80!black}\texttt{#1}}}

%%%%%%%%%
% Changed by Kushal K Dey, University of Chicago
%\providecommand{\file}[1]{}
%\renewcommand{\file}[1]{{\tt #1}}
\providecommand{\file}[1]{}
\renewcommand{\file}[1]{{\color{orange!80!black}\texttt{#1}}}
%\providecommand{\dataframe}[1]{}
%\renewcommand{\dataframe}[1]{{\color{blue!80!black}\texttt{#1}}}
\providecommand{\dataframe}[1]{}
\renewcommand{\dataframe}[1]{{\color{cyan!80!black}\texttt{#1}}}
%%%%%%%%%

% other
\def\Sum{\sum\nolimits}
\def\b#1{\fboxsep=0pt\colorbox{black}{\color{white}\Cube{#1}}}
\def\w#1{\Cube{#1}}
%%%%%%%%%%%% End of shortcuts (macros) ##############

%%%%%%%%% One way to hide answers until you want to show them %%%%%%%%%
\def\Hide#1#2{\ul{~~~\onslide<#1>{\alert{#2}}~~~}}
\def\hide#1#2{\ul{~~\onslide<#1>{\alert{#2}}~~}}
\def\hid#1#2{\onslide<#1>{\alert{#2}}}
% Choose the color of answers here too
\setbeamercolor{alerted text}{fg=darkgray} 
%\setbeamercolor{alerted text}{fg=black} 

%------Centered Page Number Setup ------
\defbeamertemplate{footline}{centered page number}
{%
  \hspace*{\fill}%
  %\usebeamercolor[fg]{page number in head/foot}%
  %\usebeamerfont{page number in head/foot}%
  \tiny \chapnum: Page \insertframenumber\, of \inserttotalframenumber%
  \hspace*{\fill}\vskip2pt%
}
%\setbeamertemplate{footline}{\hfill\insertframenumber/\inserttotalframenumber}
\setbeamertemplate{footline}[centered page number]
%--------------------------------

%\usetheme{Copenhagen}
\setbeamertemplate{navigation symbols}{}
\usepackage[english]{babel}
\def\ul{\underline}
\linespread{1.1}
% or whatever



%\parskip=0pt
\IfFileExists{upquote.sty}{\usepackage{upquote}}{}
\begin{document}%large

%<<setup, include=FALSE, cache=FALSE>>=
%options(replace.assign=TRUE,width=90, digits=4)
%opts_chunk$set(fig.path='figure/graphics-', cache.path='cache/graphics-', fig.align='center', fig.width=8, fig.height=4.5, fig.show='as.is', out.width='0.9\\linewidth', cache=FALSE, par=TRUE, size = 'tiny', tidy=TRUE, cache.extra=rand_seed)
%knit_hooks$set(par=function(before, options, envir){
%if (before && options$fig.show!='none') par(mar=c(4,4,.1,.1),cex.lab=.95,cex.axis=.9,mgp=c(2,.7,0),tcl=-.3)
%}, document = function(x) {
%  gsub('\\\\(begin|end)\\{kframe\\}', '', x)
%}, crop=hook_pdfcrop)
%@
%<<setup2, include=FALSE, cache=FALSE>>=
%knit_theme$set("print")
%@


%%%%%%%%%%%%%%%%%%%%%%%%%%%%%%%%%%%%%%%%%%%%%%%%%%%%%%%%%%%%%%%%%%%%%%%%%
%%%%%%%%%%%%%%%%%%%%%%%%%%%%%%%%%%%%%%%%%%%%%%%%%%%%%%%%%%%%%%%%%%%%%%%%%
%%%%%% End of suggested definitions and packages %%%%%%%%%%%%

%------------------------------------------------------------------
%------------------------------------------------------------------

%%%%%%%%%% Title frame (optional) %%%%%%%%%%%%%
\begin{frame}{}
\maketitle
\end{frame}
%%%%%%%%%%%%%%%%%%%%%%%%%%%%%%%%%%%%%%%%%%%%%%%

%%%%%%%%%%%%%% Begin slides here %%%%%%%%%%%%%%

%%%%%%%%%%%%%%%%%%%%%%%%%%%%%%%%%%%%%%%%%%%%%%%%%%%%%%%%%%%%
\begin{frame}[fragile]
%%%%%%%%%%%%%%%%%%%%%%%%%%%%%%%%%%%%%%%%%%%%%%%%%%%%%%%%%%%%
\frametitle{Binomial Distribution and Normal approx.}

More generally if $X_1, X_2, \cdots, X_n$ be independent identically distributed (iid) $Ber(p)$ random variables, 

then 

$$ Y_{n} = \sum_{i=1}^{n} X_{i} \sim Bin \left ( n, p \right) $$

So, 

$$ E(Y_{n}) = np  \hspace{1 cm} var(Y_{n}) = np (1-p) $$

As $n \rightarrow \infty$

$$ Y_{n} \approx N \left ( np, np(1-p) \right)  $$

\end{frame}
%%%%%%%%%%%%%%%%%%%%%%%%%%%%%%%%%%%%%%%%%%%%%%%%%%%%%%%%%%%%


%%%%%%%%%%%%%%%%%%%%%%%%%%%%%%%%%%%%%%%%%%%%%%%%%%%%%%%%%%%%
\begin{frame}[fragile]
%%%%%%%%%%%%%%%%%%%%%%%%%%%%%%%%%%%%%%%%%%%%%%%%%%%%%%%%%%%%

Lets look at variables generated at $Bin(20,0.5)$ and repeat the process $10,000$ times.

\begin{knitrout}\small
\definecolor{shadecolor}{rgb}{1, 1, 1}\color{fgcolor}\begin{kframe}
\begin{alltt}
\hlstd{Y1} \hlkwb{<-} \hlkwd{rbinom}\hlstd{(}\hlnum{10000}\hlstd{,} \hlnum{20}\hlstd{,} \hlkwc{p}\hlstd{=}\hlnum{0.5}\hlstd{)}\hlopt{/} \hlnum{20}\hlstd{;}
\hlkwd{summary}\hlstd{(Y1)}
\end{alltt}
\begin{verbatim}
   Min. 1st Qu.  Median    Mean 3rd Qu.    Max. 
  0.100   0.400   0.500   0.501   0.600   0.850 
\end{verbatim}
\end{kframe}
\end{knitrout}

\end{frame}
%%%%%%%%%%%%%%%%%%%%%%%%%%%%%%%%%%%%%%%%%%%%%%%%%%%%%%%%%%%%

%%%%%%%%%%%%%%%%%%%%%%%%%%%%%%%%%%%%%%%%%%%%%%%%%%%%%%%%%%%%
\begin{frame}[fragile]
%%%%%%%%%%%%%%%%%%%%%%%%%%%%%%%%%%%%%%%%%%%%%%%%%%%%%%%%%%%%

\begin{knitrout}\small
\definecolor{shadecolor}{rgb}{1, 1, 1}\color{fgcolor}

{\centering \includegraphics[width=0.89\linewidth]{figure/graphics-unnamed-chunk-2-1} 

}



\end{knitrout}

\end{frame}
%%%%%%%%%%%%%%%%%%%%%%%%%%%%%%%%%%%%%%%%%%%%%%%%%%%%%%%%%%%%

%%%%%%%%%%%%%%%%%%%%%%%%%%%%%%%%%%%%%%%%%%%%%%%%%%%%%%%%%%%%
\begin{frame}[fragile]
%%%%%%%%%%%%%%%%%%%%%%%%%%%%%%%%%%%%%%%%%%%%%%%%%%%%%%%%%%%%

Now lets look at variables generated at $Bin(100,0.5)$ and repeat the process $10,000$ times.

\begin{knitrout}\small
\definecolor{shadecolor}{rgb}{1, 1, 1}\color{fgcolor}\begin{kframe}
\begin{alltt}
\hlstd{Y2} \hlkwb{<-} \hlkwd{rbinom}\hlstd{(}\hlnum{10000}\hlstd{,} \hlnum{100}\hlstd{,} \hlkwc{p}\hlstd{=}\hlnum{0.5}\hlstd{)}\hlopt{/} \hlnum{100}\hlstd{;}
\hlkwd{summary}\hlstd{(Y2)}
\end{alltt}
\begin{verbatim}
   Min. 1st Qu.  Median    Mean 3rd Qu.    Max. 
   0.31    0.47    0.50    0.50    0.53    0.67 
\end{verbatim}
\end{kframe}
\end{knitrout}

\end{frame}
%%%%%%%%%%%%%%%%%%%%%%%%%%%%%%%%%%%%%%%%%%%%%%%%%%%%%%%%%%%%

%%%%%%%%%%%%%%%%%%%%%%%%%%%%%%%%%%%%%%%%%%%%%%%%%%%%%%%%%%%%
\begin{frame}[fragile]
%%%%%%%%%%%%%%%%%%%%%%%%%%%%%%%%%%%%%%%%%%%%%%%%%%%%%%%%%%%%

\begin{knitrout}\small
\definecolor{shadecolor}{rgb}{1, 1, 1}\color{fgcolor}

{\centering \includegraphics[width=0.89\linewidth]{figure/graphics-unnamed-chunk-4-1} 

}



\end{knitrout}

\end{frame}
%%%%%%%%%%%%%%%%%%%%%%%%%%%%%%%%%%%%%%%%%%%%%%%%%%%%%%%%%%%%

%%%%%%%%%%%%%%%%%%%%%%%%%%%%%%%%%%%%%%%%%%%%%%%%%%%%%%%%%%%%
\begin{frame}[fragile]
%%%%%%%%%%%%%%%%%%%%%%%%%%%%%%%%%%%%%%%%%%%%%%%%%%%%%%%%%%%%

Now lets look at variables generated at $Bin(1000,0.5)$ and repeat the process $10,000$ times.

\begin{knitrout}\small
\definecolor{shadecolor}{rgb}{1, 1, 1}\color{fgcolor}\begin{kframe}
\begin{alltt}
\hlstd{Y3} \hlkwb{<-} \hlkwd{rbinom}\hlstd{(}\hlnum{10000}\hlstd{,} \hlnum{1000}\hlstd{,} \hlkwc{p}\hlstd{=}\hlnum{0.5}\hlstd{)}\hlopt{/} \hlnum{1000}\hlstd{;}
\hlkwd{summary}\hlstd{(Y3)}
\end{alltt}
\begin{verbatim}
   Min. 1st Qu.  Median    Mean 3rd Qu.    Max. 
  0.442   0.489   0.500   0.500   0.511   0.561 
\end{verbatim}
\end{kframe}
\end{knitrout}

\end{frame}
%%%%%%%%%%%%%%%%%%%%%%%%%%%%%%%%%%%%%%%%%%%%%%%%%%%%%%%%%%%%

%%%%%%%%%%%%%%%%%%%%%%%%%%%%%%%%%%%%%%%%%%%%%%%%%%%%%%%%%%%%
\begin{frame}[fragile]
%%%%%%%%%%%%%%%%%%%%%%%%%%%%%%%%%%%%%%%%%%%%%%%%%%%%%%%%%%%%

\begin{knitrout}\small
\definecolor{shadecolor}{rgb}{1, 1, 1}\color{fgcolor}

{\centering \includegraphics[width=0.89\linewidth]{figure/graphics-unnamed-chunk-6-1} 

}



\end{knitrout}

\end{frame}
%%%%%%%%%%%%%%%%%%%%%%%%%%%%%%%%%%%%%%%%%%%%%%%%%%%%%%%%%%%%


%%%%%%%%%%%%%%%%%%%%%%%%%%%%%%%%%%%%%%%%%%%%%%%%%%%%%%%%%%%%
\begin{frame}[fragile]
%%%%%%%%%%%%%%%%%%%%%%%%%%%%%%%%%%%%%%%%%%%%%%%%%%%%%%%%%%%%
\frametitle{Normal approx. of Binomial}

\url{http://digitalfirst.bfwpub.com/stats_applet/stats_applet_2_cltbinom.html}

We define $ \hat{p} = \frac{1}{n} \sum_{i=1}^{n} X_{i}$.

$$ \hat{p} \approx N \left (p, \frac{p(1-p)}{n} \right) $$

This approximation is better when $p$ is not to close to $0$ or $1$.

\end{frame}
%%%%%%%%%%%%%%%%%%%%%%%%%%%%%%%%%%%%%%%%%%%%%%%%%%%%%%%%%%%%

%%%%%%%%%%%%%%%%%%%%%%%%%%%%%%%%%%%%%%%%%%%%%%%%%%%%%%%%%%%%
\begin{frame}[fragile]
%%%%%%%%%%%%%%%%%%%%%%%%%%%%%%%%%%%%%%%%%%%%%%%%%%%%%%%%%%%%
\frametitle{Continuity Correction}

\begin{tabular}{|c|c|}
\hline
Discrete & Continuous \\ \hline
$X=3$ & $2.5 < X < 3.5$ \\ \hline
$X > 3$ & $X > 3.5$ \\ \hline
$X \geq 3$ & $ X > 2.5 $ \\ \hline
$X < 3$ & $X < 2.5$ \\ \hline
$X \leq 3$ & $X < 3.5$ \\ \hline
\end{tabular}

\end{frame}
%%%%%%%%%%%%%%%%%%%%%%%%%%%%%%%%%%%%%%%%%%%%%%%%%%%%%%%%%%%%

%%%%%%%%%%%%%%%%%%%%%%%%%%%%%%%%%%%%%%%%%%%%%%%%%%%%%%%%%%%%
\begin{frame}[fragile]
%%%%%%%%%%%%%%%%%%%%%%%%%%%%%%%%%%%%%%%%%%%%%%%%%%%%%%%%%%%%

\frametitle{Continuity Correction}

\begin{itemize}
\item $X$ is approximately Normal with mean $np$ and sd $\sqrt{np(1-p)}$ \pause
\item But $X$ is a discrete random variable and Normal is a continuous one. \pause
\item $X \sim Bin(16, 0.6 ) $. How to find $P( X \leq 3)$? \pause $P(X \leq 3)= P( X \leq 3.5)$ ? \pause
\item By R : \textit{ pbinom(3,16,0.6) \# 0.000938 } \pause
\item $$P( X \leq 3 )= P( \frac{X- 9.6}{1.959} \leq \frac{3-9.6}{1.959}) \pause = P(z \leq -3.369) = 0.00037$$ \pause
\item $$P( X < 3.5 )= P( \frac{X- 9.6}{1.959} \leq \frac{3.5-9.6}{1.959})  = P(z \leq -3.114) = 0.000923$$ \pause
\item To find $P(X \geq 10)$ Should we use 10.5 or 9.5 ? 
\end{itemize}

\end{frame}
%%%%%%%%%%%%%%%%%%%%%%%%%%%%%%%%%%%%%%%%%%%%%%%%%%%%%%%%%%%%

%%%%%%%%%%%%%%%%%%%%%%%%%%%%%%%%%%%%%%%%%%%%%%%%%%%%%%%%%%%%
\begin{frame}[fragile]
%%%%%%%%%%%%%%%%%%%%%%%%%%%%%%%%%%%%%%%%%%%%%%%%%%%%%%%%%%%%

\frametitle{Continuous distribution}

We would want to define $Pr(X=x)$ for a continuous random variable $X$, but unfortunately it is $0$. We will soon see why \pause \newline

So how do we proceed to build something like a probability table for discrete variable?

We define \textbf{cumulative density function} (cdf)

$$ F(x) = Pr (X \leq x) $$

If we differentiate this function, we get \textbf{probability density function} (pdf)

$$ \frac{d}{dx} F(x) = f(x) $$

$$ f(x) \geq 0 \hspace {1 cm} \int f(x) dx = 1 $$

\end{frame}
%%%%%%%%%%%%%%%%%%%%%%%%%%%%%%%%%%%%%%%%%%%%%%%%%%%%%%%%%%%%

%%%%%%%%%%%%%%%%%%%%%%%%%%%%%%%%%%%%%%%%%%%%%%%%%%%%%%%%%%%%
\begin{frame}[fragile]
%%%%%%%%%%%%%%%%%%%%%%%%%%%%%%%%%%%%%%%%%%%%%%%%%%%%%%%%%%%%

\frametitle{Normal Distribution}

For normal distribution with mean $\mu$ and variance $\sigma^2$, 

$$ \phi(x) := \int \frac{1}{\sqrt{2 \pi}} exp \left (-\frac{(x-\mu)^2}{2 \sigma^2} \right ) $$

Cumulative distribution

$$ \Phi(x) : = \int_{-\infty}^{x} \phi(x) dx $$

The mean and variance 

$$ E(X) = \mu  \hspace{1 cm} var(X) = \sigma^2 $$

\end{frame}
%%%%%%%%%%%%%%%%%%%%%%%%%%%%%%%%%%%%%%%%%%%%%%%%%%%%%%%%%%%%

%%%%%%%%%%%%%%%%%%%%%%%%%%%%%%%%%%%%%%%%%%%%%%%%%%%%%%%%%%%%
\begin{frame}{probability density graph}
%%%%%%%%%%%%%%%%%%%%%%%%%%%%%%%%%%%%%%%%%%%%%%%%%%%%%%%%%%%%

\begin{knitrout}\small
\definecolor{shadecolor}{rgb}{1, 1, 1}\color{fgcolor}

{\centering \includegraphics[width=0.99\linewidth]{figure/graphics-unnamed-chunk-7-1} 

}



\end{knitrout}

\end{frame}
%%%%%%%%%%%%%%%%%%%%%%%%%%%%%%%%%%%%%%%%%%%%%%%%%%%%%%%%%%%%

%%%%%%%%%%%%%%%%%%%%%%%%%%%%%%%%%%%%%%%%%%%%%%%%%%%%%%%%%%%%
\begin{frame}{cumulative density graph}
%%%%%%%%%%%%%%%%%%%%%%%%%%%%%%%%%%%%%%%%%%%%%%%%%%%%%%%%%%%%

\begin{knitrout}\small
\definecolor{shadecolor}{rgb}{1, 1, 1}\color{fgcolor}

{\centering \includegraphics[width=0.99\linewidth]{figure/graphics-unnamed-chunk-8-1} 

}



\end{knitrout}

\end{frame}
%%%%%%%%%%%%%%%%%%%%%%%%%%%%%%%%%%%%%%%%%%%%%%%%%%%%%%%%%%%%

%%%%%%%%%%%%%%%%%%%%%%%%%%%%%%%%%%%%%%%%%%%%%%%%%%%%%%%%%%%%
\begin{frame}
%%%%%%%%%%%%%%%%%%%%%%%%%%%%%%%%%%%%%%%%%%%%%%%%%%%%%%%%%%%%
\frametitle{Sum of normal variables}

Suppose we consider two random variables $X$ and $Y$ which follow normal distribution.

$$ X \sim N(\mu, \sigma^2)  \hspace{1 cm} Y \sim N(\mu, \sigma^2) $$ \pause \newline

Assume now that $X$ and $Y$ are independent. \pause \newline

What is the distribution of $X+Y$ \pause \newline 

$$ E(X+Y) = E(X) + E(Y) = \mu + \mu = 2 \mu $$
$$ var(X+Y) = var(X) + var(Y) = \sigma^2 + \sigma^2 = 2 \sigma^2 $$

\end{frame}
%%%%%%%%%%%%%%%%%%%%%%%%%%%%%%%%%%%%%%%%%%%%%%%%%%%%%%%%%%%%

%%%%%%%%%%%%%%%%%%%%%%%%%%%%%%%%%%%%%%%%%%%%%%%%%%%%%%%%%%%%
\begin{frame}[fragile]
%%%%%%%%%%%%%%%%%%%%%%%%%%%%%%%%%%%%%%%%%%%%%%%%%%%%%%%%%%%%
\frametitle{Sum of normal variables}

To find the distribution, we perform a large number of repetitions (close to infinity)
of the experiment of drawing random variables $X$ and $Y$. \pause 

Suppose we repeat it $100,000$ times. \pause

\begin{knitrout}\small
\definecolor{shadecolor}{rgb}{1, 1, 1}\color{fgcolor}\begin{kframe}
\begin{alltt}
\hlstd{x} \hlkwb{<-} \hlkwd{rnorm}\hlstd{(}\hlnum{100000}\hlstd{,} \hlnum{1}\hlstd{,} \hlnum{1}\hlstd{);}
\hlstd{y} \hlkwb{<-} \hlkwd{rnorm}\hlstd{(}\hlnum{100000}\hlstd{,} \hlnum{1}\hlstd{,} \hlnum{1}\hlstd{);}
\hlstd{z} \hlkwb{<-} \hlstd{x}\hlopt{+}\hlstd{y;}
\hlkwd{length}\hlstd{(z)}
\end{alltt}
\begin{verbatim}
[1] 100000
\end{verbatim}
\end{kframe}
\end{knitrout}

\end{frame}
%%%%%%%%%%%%%%%%%%%%%%%%%%%%%%%%%%%%%%%%%%%%%%%%%%%%%%%%%%%%

%%%%%%%%%%%%%%%%%%%%%%%%%%%%%%%%%%%%%%%%%%%%%%%%%%%%%%%%%%%%
\begin{frame}[fragile]
%%%%%%%%%%%%%%%%%%%%%%%%%%%%%%%%%%%%%%%%%%%%%%%%%%%%%%%%%%%%

\frametitle{Sum of normal variables}

\begin{knitrout}\small
\definecolor{shadecolor}{rgb}{1, 1, 1}\color{fgcolor}

{\centering \includegraphics[width=0.99\linewidth]{figure/graphics-unnamed-chunk-10-1} 

}



\end{knitrout}

\end{frame}
%%%%%%%%%%%%%%%%%%%%%%%%%%%%%%%%%%%%%%%%%%%%%%%%%%%%%%%%%%%%


%%%%%%%%%%%%%%%%%%%%%%%%%%%%%%%%%%%%%%%%%%%%%%%%%%%%%%%%%%%%
\begin{frame}[fragile]
%%%%%%%%%%%%%%%%%%%%%%%%%%%%%%%%%%%%%%%%%%%%%%%%%%%%%%%%%%%%
\frametitle{Sum of normal variables}

Assume now three random variables $X$, $Y$ and $W$ and consider their sum 

$$  Z = X+Y+W $$ \pause  \newline

Suppose we repeat it $100,000$ times. \pause

\begin{knitrout}\small
\definecolor{shadecolor}{rgb}{1, 1, 1}\color{fgcolor}\begin{kframe}
\begin{alltt}
\hlstd{x} \hlkwb{<-} \hlkwd{rnorm}\hlstd{(}\hlnum{100000}\hlstd{,} \hlnum{1}\hlstd{,} \hlnum{1}\hlstd{);}
\hlstd{y} \hlkwb{<-} \hlkwd{rnorm}\hlstd{(}\hlnum{100000}\hlstd{,} \hlnum{1}\hlstd{,} \hlnum{3}\hlstd{);}
\hlstd{w} \hlkwb{<-} \hlkwd{rnorm}\hlstd{(}\hlnum{100000}\hlstd{,} \hlnum{10}\hlstd{,} \hlnum{2}\hlstd{);}
\hlstd{z} \hlkwb{<-} \hlstd{x}\hlopt{+}\hlstd{y}\hlopt{+}\hlstd{w;}
\hlkwd{length}\hlstd{(z)}
\end{alltt}
\begin{verbatim}
[1] 100000
\end{verbatim}
\end{kframe}
\end{knitrout}

\end{frame}
%%%%%%%%%%%%%%%%%%%%%%%%%%%%%%%%%%%%%%%%%%%%%%%%%%%%%%%%%%%%

%%%%%%%%%%%%%%%%%%%%%%%%%%%%%%%%%%%%%%%%%%%%%%%%%%%%%%%%%%%%
\begin{frame}[fragile]
%%%%%%%%%%%%%%%%%%%%%%%%%%%%%%%%%%%%%%%%%%%%%%%%%%%%%%%%%%%%

\frametitle{Sum of normal variables}

\begin{knitrout}\small
\definecolor{shadecolor}{rgb}{1, 1, 1}\color{fgcolor}

{\centering \includegraphics[width=0.99\linewidth]{figure/graphics-unnamed-chunk-12-1} 

}



\end{knitrout}

\end{frame}
%%%%%%%%%%%%%%%%%%%%%%%%%%%%%%%%%%%%%%%%%%%%%%%%%%%%%%%%%%%%


%%%%%%%%%%%%%%%%%%%%%%%%%%%%%%%%%%%%%%%%%%%%%%%%%%%%%%%%%%%%
\begin{frame}[fragile]
%%%%%%%%%%%%%%%%%%%%%%%%%%%%%%%%%%%%%%%%%%%%%%%%%%%%%%%%%%%%

\frametitle{Result}

If $X_1$, $X_2$, $\cdots$, $X_n$ are \emph{independent} normal random variables such that 

$$ X_{i} \sim N \left (\mu_i, \sigma^2_{i} \right) $$

Then if we define 

$$ Z = X_1 + X_2 + \cdots + X_{n} $$

then 

$$  Z \sim N \left (\sum_{i=1}^{n} \mu_{i}, \sum_{i=1}^{n} \sigma^2_{i} \right ) $$

\end{frame}
%%%%%%%%%%%%%%%%%%%%%%%%%%%%%%%%%%%%%%%%%%%%%%%%%%%%%%%%%%%%

%%%%%%%%%%%%%%%%%%%%%%%%%%%%%%%%%%%%%%%%%%%%%%%%%%%%%%%%%%%%
\begin{frame}[fragile]
%%%%%%%%%%%%%%%%%%%%%%%%%%%%%%%%%%%%%%%%%%%%%%%%%%%%%%%%%%%%
\frametitle{Scaling of normal variables}

Let $X$ be a random variable that follows a distribution 

$$ X \sim N(\mu, \sigma^2) $$

Let $a$ be a constant.

What is the distribution of $aX$.

\begin{knitrout}\small
\definecolor{shadecolor}{rgb}{1, 1, 1}\color{fgcolor}\begin{kframe}
\begin{alltt}
\hlstd{x} \hlkwb{<-} \hlkwd{rnorm}\hlstd{(}\hlnum{100000}\hlstd{,} \hlnum{1}\hlstd{,} \hlnum{1}\hlstd{);}
\hlstd{a} \hlkwb{<-} \hlnum{4}\hlstd{;}
\hlstd{z} \hlkwb{<-} \hlstd{a}\hlopt{*}\hlstd{x}
\hlkwd{length}\hlstd{(z)}
\end{alltt}
\begin{verbatim}
[1] 100000
\end{verbatim}
\end{kframe}
\end{knitrout}

\end{frame}
%%%%%%%%%%%%%%%%%%%%%%%%%%%%%%%%%%%%%%%%%%%%%%%%%%%%%%%%%%%%

%%%%%%%%%%%%%%%%%%%%%%%%%%%%%%%%%%%%%%%%%%%%%%%%%%%%%%%%%%%%
\begin{frame}[fragile]
%%%%%%%%%%%%%%%%%%%%%%%%%%%%%%%%%%%%%%%%%%%%%%%%%%%%%%%%%%%%

\frametitle{Sum of normal variables}

\begin{knitrout}\small
\definecolor{shadecolor}{rgb}{1, 1, 1}\color{fgcolor}

{\centering \includegraphics[width=0.99\linewidth]{figure/graphics-unnamed-chunk-14-1} 

}



\end{knitrout}

\end{frame}
%%%%%%%%%%%%%%%%%%%%%%%%%%%%%%%%%%%%%%%%%%%%%%%%%%%%%%%%%%%%

%%%%%%%%%%%%%%%%%%%%%%%%%%%%%%%%%%%%%%%%%%%%%%%%%%%%%%%%%%%%
\begin{frame}[fragile]
%%%%%%%%%%%%%%%%%%%%%%%%%%%%%%%%%%%%%%%%%%%%%%%%%%%%%%%%%%%%

\frametitle{Result}

If $X$ be a  normal random variables such that 

$$ X \sim N \left (\mu, \sigma^2 \right) $$

Let $a$ be a constant,

$$ Z = aX $$

$$ E(Z) = aE(X) = a\mu $$

$$ var(Z)= var(aX) = a^2 var(X) = a^2 \sigma^2 $$

and 

$$  Z \sim N \left (a\mu, a^2\sigma^2 \right ) $$

\end{frame}
%%%%%%%%%%%%%%%%%%%%%%%%%%%%%%%%%%%%%%%%%%%%%%%%%%%%%%%%%%%%

%%%%%%%%%%%%%%%%%%%%%%%%%%%%%%%%%%%%%%%%%%%%%%%%%%%%%%%%%%%%
\begin{frame}[fragile]
%%%%%%%%%%%%%%%%%%%%%%%%%%%%%%%%%%%%%%%%%%%%%%%%%%%%%%%%%%%%

\frametitle{Result}

If $X_1$, $X_2$, $\cdots$, $X_n$ are \emph{independent} normal random variables such that 

$$ X_{i} \sim N \left (\mu_i, \sigma^2_{i} \right) $$

Then if we define 

$$ Z = c_1 X_1 + c_2 X_2 + \cdots + c_n X_{n} $$

Check 

$$ E(Z) = \sum_{i=1}^{n} c_{i} \mu_{i} \hspace{1 cm} var(Z)=\sum_{i=1}^{n} c^2_{i} \sigma^2_{i} $$

then 

$$  Z \sim N \left (\sum_{i=1}^{n} c_{i} \mu_{i}, \sum_{i=1}^{n} c^2_{i} \sigma^2_{i} \right) $$

\end{frame}
%%%%%%%%%%%%%%%%%%%%%%%%%%%%%%%%%%%%%%%%%%%%%%%%%%%%%%%%%%%%

%%%%%%%%%%%%%%%%%%%%%%%%%%%%%%%%%%%%%%%%%%%%%%%%%%%%%%%%%%%%
\begin{frame}[fragile]
%%%%%%%%%%%%%%%%%%%%%%%%%%%%%%%%%%%%%%%%%%%%%%%%%%%%%%%%%%%%

\frametitle{Corollary of Previous Result}

If $X_1$, $X_2$, $\cdots$, $X_n$ are \emph{independent} normal random variables such that 

$$ X_{i} \sim N \left (\mu, \sigma^2 \right) $$

then if we define 

$$ Z = \frac{1}{n} \sum_{i=1}^{n} X_{i} $$

and 

$$  Z \sim N \left (\mu, \frac{\sigma^2}{n} \right) $$

\end{frame}
%%%%%%%%%%%%%%%%%%%%%%%%%%%%%%%%%%%%%%%%%%%%%%%%%%%%%%%%%%%%

%%%%%%%%%%%%%%%%%%%%%%%%%%%%%%%%%%%%%%%%%%%%%%%%%%%%%%%%%%%%
\begin{frame}[fragile]
%%%%%%%%%%%%%%%%%%%%%%%%%%%%%%%%%%%%%%%%%%%%%%%%%%%%%%%%%%%%

\frametitle{Conclusions}

We showed that sum of independent normal random variables is normal. \pause \newline

Linear transformation of normal random variables is normal. \pause \newline

Sum of independent Bernoulli random variables is Binomial. \pause \newline

Sum of independent Binomial random variables?  \pause \newline

Its Binomial.

\end{frame}
%%%%%%%%%%%%%%%%%%%%%%%%%%%%%%%%%%%%%%%%%%%%%%%%%%%%%%%%%%%%

%%%%%%%%%%%%%%%%%%%%%%%%%%%%%%%%%%%%%%%%%%%%%%%%%%%%%%%%%%%%
\begin{frame}[fragile]
%%%%%%%%%%%%%%%%%%%%%%%%%%%%%%%%%%%%%%%%%%%%%%%%%%%%%%%%%%%%

\frametitle{Central Limit Theorem}

All the results we discussed today are true for sum of any $n$ independent variables,
where $n$ can be small or large. What about large n? \pause \newline

if $X_1, X_2, \cdots, X_n$ be independent identically distributed (iid) random variables coming from a distribution with mean $\mu$ and variance $\sigma^2$, 

then 

$$ \sum_{i=1}^{n} X_{i} \approx N \left (n \mu, n \sigma^2 \right )  \hspace{0.5 cm} n \;\; large $$

and 

$$ \frac{1}{n}\sum_{i=1}^{n} X_{i} \approx N \left (\mu, \frac{\sigma^2}{n} \right ) \hspace{0.5 cm} n \;\; large$$

\end{frame}
%%%%%%%%%%%%%%%%%%%%%%%%%%%%%%%%%%%%%%%%%%%%%%%%%%%%%%%%%%%%

%%%%%%%%%%%%%%%%%%%%%%%%%%%%%%%%%%%%%%%%%%%%%%%%%%%%%%%%%%%%
\begin{frame}[fragile]
%%%%%%%%%%%%%%%%%%%%%%%%%%%%%%%%%%%%%%%%%%%%%%%%%%%%%%%%%%%%

\frametitle{Conclusions}

CLT claims sum of a large number of random variables coming from any distribution (well behaved) is approximately normal. \pause \newline

As a special case, Sum of a large number of Bernoulli or Binomial random variables is approximately normal. \pause \newline

When the underlying distribution is discrete, remember \textit{continuity correction}.

\end{frame}
%%%%%%%%%%%%%%%%%%%%%%%%%%%%%%%%%%%%%%%%%%%%%%%%%%%%%%%%%%%%

%%%%%%%%%%%%%%%%%%%%%%%%%%%%%%%%%%%%%%%%%%%%%%%%%%%%%%%%%%%%
\begin{frame}[fragile]
%%%%%%%%%%%%%%%%%%%%%%%%%%%%%%%%%%%%%%%%%%%%%%%%%%%%%%%%%%%%

\frametitle{Moment generating function}

We define a moment generating function (mgf) as a function of $t$ 

$$ mgf(t) : = E \left (e^{tX} \right)  = \int e^{tx} f(x) dx $$

where $f(x)$ is the probability density function (pdf) observed at point $x$. \pause \newline

\textbf{Very important note}: Moment generating functions characterize distributions for most cases. \pause \newline

What does that mean? \pause \newline

\end{frame}
%%%%%%%%%%%%%%%%%%%%%%%%%%%%%%%%%%%%%%%%%%%%%%%%%%%%%%%%%%%%

%%%%%%%%%%%%%%%%%%%%%%%%%%%%%%%%%%%%%%%%%%%%%%%%%%%%%%%%%%%%
\begin{frame}[fragile]
%%%%%%%%%%%%%%%%%%%%%%%%%%%%%%%%%%%%%%%%%%%%%%%%%%%%%%%%%%%%

\frametitle{Normal Moment generating function}

For a normal random variable $X$, it can be shown that the moment generating function has the following form 

$$ mgf(t) : = exp \left (\mu t + \frac{1}{2} t^2 \sigma^2 \right ) $$

Using the characterizing property of mgf, if suppose we have 

$$ mgf(t): exp \left (2 t + \frac{1}{2} 6t^2\right ) $$

then this is the mgf of 

$$ X \sim N(2,6) $$

\end{frame}
%%%%%%%%%%%%%%%%%%%%%%%%%%%%%%%%%%%%%%%%%%%%%%%%%%%%%%%%%%%%

%%%%%%%%%%%%%%%%%%%%%%%%%%%%%%%%%%%%%%%%%%%%%%%%%%%%%%%%%%%%
\begin{frame}[fragile]
%%%%%%%%%%%%%%%%%%%%%%%%%%%%%%%%%%%%%%%%%%%%%%%%%%%%%%%%%%%%

\frametitle{Moment generating function for sums}

If $X_1$, $X_2$, $\cdots$, $X_{n}$ are independent random variables following a distribution with pdf $f(x)$, then the moment generating function of $X$

$$ mgf_{X}(t) : E(e^{tX})  $$

If we define 

$$ Y = X_1 + X_2 + \cdots X_n $$

$$ mgf_{Y}(t):  E(e^{tY}) = E(e^{tX_1 + tX_2 + \cdots + tX_n}) $$

We can show that  (check Canvas for proof)

$$ mgf_{Y}(t) = \left [ mgf_{X}(t) \right ]^{n} $$

\end{frame}
%%%%%%%%%%%%%%%%%%%%%%%%%%%%%%%%%%%%%%%%%%%%%%%%%%%%%%%%%%%%

%%%%%%%%%%%%%%%%%%%%%%%%%%%%%%%%%%%%%%%%%%%%%%%%%%%%%%%%%%%%
\begin{frame}[fragile]
%%%%%%%%%%%%%%%%%%%%%%%%%%%%%%%%%%%%%%%%%%%%%%%%%%%%%%%%%%%%


\frametitle{How to use mgf}

We use the mgf properties to show that sum of normal random variables is normal, or sum of linear transformation of normal random variables is normal. \pause \newline

Check Canvas for the detailed proof. We will give a brief sketch here. \pause \newline

Questions?  \pause \newline

\end{frame}
%%%%%%%%%%%%%%%%%%%%%%%%%%%%%%%%%%%%%%%%%%%%%%%%%%%%%%%%%%%%




\end{document}
